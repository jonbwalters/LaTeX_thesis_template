
\chapter{INTRODUCTION}\label{chap1:introduction}

{\bf Put your introduction here.}


Typesetting in \LaTeX \ is {\em the} way to communicate mathematics
(journal articles, books, MS theses, doctoral dissertations),
and \LaTeX \ is also popular in other disciplines, such as 
physics or computer science. This template should make the formatting
of your thesis easier. It is set up so that, unless there are some really
wide equations or figures, all margins should automatically be correct.
Similarly, all parts are in the order required by the graduate school and 
the required tables (listing contents, figures, tables)
are created automatically in the required format. 
The template is compiled by running \LaTeX \ on 
the file \verb+thesis.tex+. The only part of 
\verb+thesis.tex+ that you should change are the
\verb+\include{...}+ commands: Add more to accommodate all the
parts of your thesis and comment out those that do not apply. 


For the actual software,
\begin{itemize}
	\item Mik\TeX~\cite{MiKTeX}
	is a standard \LaTeX\ distribution for Windows and
	WinEdt~\cite{WinEdt} is a standard front end.
	WinEdt asks for a registration fee. 
	The similar \TeX nicCenter~\cite{TeXnicCenter}
	is free. 
	\item Mac\TeX\ is the standard MacOS distribution.
	It comes with the \TeX Shop editor.
	\item \TeX Live is the standard Linux distribution.
	\item Overleaf is a convenient web app for hosting your
	\TeX projects.
\end{itemize}

A more thorough introduction to contemporary \LaTeX\ can be
found in the \LaTeX\ Wikibook.