% The preamble contains macro definitions





\def\qed{\hfill\ \rule{2mm}{2mm} }



\documentclass[12pt]{article}



%  The newtheorem ``guess" is used to get a uniform numbering of all items
%    in each section.


\newtheorem{guess}{Guess}[section]

\newtheorem{define}[guess]{Definition}

\newtheorem{prop}[guess]{Proposition}

\newtheorem{theorem}[guess]{Theorem}

\newtheorem{cor}[guess]{Corollary}

\newtheorem{lem}[guess]{Lemma}






\begin{document}




\title{
Title Goes Here
\thanks{Acknowledgements can go here.} }



\author{
\small Name\\
\small Address 1\\
\small Address 2\\
\small Address 3\\
\small {\tt e-mail }
}


\date{\small \today}

\maketitle


\begin{abstract}

Abstract goes here.

This template is intended to demonstrate/make available some of the most used
features of \LaTeX. These are the sectioning, enumeration and referencing features as
well as the equation array. In Section \ref{explore} some possible explorations of
the automatic renumbering are suggested.

I use this template for all my papers. B. Schr\"oder.

\end{abstract}


\section{Section Title}

\begin{theorem}
\label{th1}

Let's say this is our first theorem.

\end{theorem}



\begin{define}
\label{def1}

Let's say this is our first definition.

\end{define}


Here's how mathematics can be displayed.


\begin{itemize}
\item
In-text math mode: $\int _0 ^1 x^2 dx = {1\over 3} .$



\item
Display  math mode: $$\int _0 ^1 x^2 dx = {1\over 3} .$$

\item
Display within text math mode: $\displaystyle{ \int _0 ^1 x^2 dx = {1\over 3} .} $

\item
And here is how we line up equalities (equation array):

\begin{eqnarray*}
\int _0 ^1 x^2 dx
& = &
\left. {1\over 3} x^2 \right| _{0} ^1
\\
& = &
{1\over 3}
\end{eqnarray*}

Note that delimiters like braces can be automatically adjusted to the right size
by using the ``$\setminus $left" and ``$\setminus $right" commands.
The above shows that the delimiters themselves need not match in shape and
a period gives an unprinted dummy delimiter.

\end{itemize}




\section{Next Section Title}



\begin{define}
\label{def2}

Let's say this is our second definition.

\end{define}


\begin{theorem}
\label{th2}

Let's say this is our second theorem.

\end{theorem}



The block at the end of a proof can be generated like this. \qed


\vspace{.1in}

Enumeration is done like this

\begin{enumerate}
\item
\label{en1}
First entry.

\item
\label{en2}
Second entry.

\end{enumerate}





\section{Things to Explore}
\label{explore}

Go ahead and move the theorems, definitions and items around.
After compiling twice, the references will be correct.

\begin{itemize}
\item
The first theorem is Theorem \ref{th1}.

\item
The second theorem is Theorem \ref{th2}.

\item
The first definition is Definition \ref{def1}.

\item
The second definition is Definition \ref{def2}.

\item
The first enumeration entry is \ref{en1}.

\item
The second enumeration entry is \ref{en2}.


\item
The first entry in the bibliography is \cite{bib1}.

\item
The second entry in the bibliography is \cite{bib2}.

\end{itemize}



\begin{thebibliography}{99}
\bibitem{bib1}
First entry in the bibliography.

\bibitem{bib2}
Second entry in the bibliography.


\end{thebibliography}





\end{document}
