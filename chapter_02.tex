\chapter{TYPESETTING IN \LaTeX }\label{chap2:latex}


You can see how the various types of environments
are created by looking at the corresponding source code.


\section{Section Title}

\begin{theorem}
\label{th1}

Let's say this is our first theorem.

\end{theorem}


{\bf Proof.}
Type proofs as regular paragraphs. \LaTeX \ will treat them as
regular paragraphs and automatically format them consistently.


\begin{definition}
\label{def1}

Let's say this is our first definition.

\end{definition}


Here's how mathematics can be displayed.
Note that a one-line ``paragraph" before an
itemized or enumerated list can look a bit funny, because the
indentations don't match up.

\begin{itemize}
\item
In-text math mode: $\int _0 ^1 x^2 dx = {1\over 3} .$



\item
Display  math mode:
{\em
To assure the proper spacing, the \$ \$ $\cdots $ \$ \$
environment
must start on the line {\bf immediately} after the preceding text.}
$$\int _0 ^1 x^2 dx = {1\over 3} .$$

\item
Display within text math mode: $\displaystyle{ \int _0 ^1 x^2 dx = {1\over 3} .} $

\item
And here is how we line up equalities (equation array).
{\em To assure the proper spacing, the $\setminus $begin$\{ $eqnarray*$\} $
must be on the line {\bf immediately} after the preceding text.}
\begin{eqnarray*}
\int _0 ^1 x^2 dx
& = &
\left. {1\over 3} x^2 \right| _{0} ^1
\\
& = &
{1\over 3}
\end{eqnarray*}

\end{itemize}


Note that delimiters like braces can be automatically adjusted to the right size
by using the ``$\setminus $left" and ``$\setminus $right" commands.
The above shows that the delimiters themselves need not match in shape and
that a period gives an unprinted dummy delimiter.




The equation array also works when things get ugly:
\begin{eqnarray*}
\lefteqn{
\int _{\Omega
} \left|
D^\alpha f(x)
-D^\alpha f_a (x)\right| ^p
\ d\lambda (x)
}
\\
& = &
\int _{\Omega
}
\left| \int _\Omega D^\alpha f(x) g_a (x-z) \ d\lambda (z)-
\int _\Omega D^\alpha f (z) g_a (x-z)  \ d\lambda (z)
\right| ^p
\ d\lambda (x)
\\
& \leq &
\int _\Omega \left(
\int _\Omega \left| D^\alpha f(x)-D^\alpha f (z) \right|
g_a (x-z)  \ d\lambda (z)
\right) ^p
\ d\lambda (x)
\\
& \leq &
\int _\Omega \left(
\left(
\int _\Omega \left( \left| D^\alpha f(x)-D^\alpha f (z) \right|
\big( g_a (x-z)\big) ^{1\over p}  \right) ^p \ d\lambda (z)
\right) ^{1\over p}
\times \right.
\\
& &
\quad
\times
\left.
\left(
\int _\Omega
\left(
\big( g_a (x-z)\big) ^{1-{1\over p}}  \right) ^q \ d\lambda (z)
\right) ^{1\over q}
\right) ^p
\ d\lambda (x)
\\
& = &
\int _\Omega
\int _\Omega \left| D^\alpha f(x)-D^\alpha f (z) \right| ^p
g_a (x-z) \ d\lambda (z)
\left(
\int _\Omega
g_a (x-z)   \ d\lambda (z)
\right) ^{p\over q}
\ d\lambda (x)
\\
& \leq &
\int _\Omega
\int _{B_a (0)} \left| D^\alpha f(x)-D^\alpha f (x+y) \right| ^p
g_a (-y) \ d\lambda (y)
\ d\lambda (x)
\\
& = &
\int _{B_a (0)}
\int _\Omega
\left| D^\alpha f(x)-D^\alpha f (x+y) \right| ^p
\ d\lambda (x)
g_a (y) \ d\lambda (y)
\\
& < &
\int _{B_a (0)}
\nu
g_a (y) \ d\lambda (y)
=\nu .
\end{eqnarray*}





\section{Next Section Title}



\begin{definition}
\label{def2}

Let's say this is our second definition.

\end{definition}


\begin{theorem}
\label{th2}

Let's say this is our second theorem.

\end{theorem}



The block at the end of a proof can be generated like this. \qed


\section{Enumeration and Itemization}

{\bf Enumeration} is done like this

\begin{enumerate}
\item
\label{en1}
First entry.

\item
\label{en2}
Second entry.

\end{enumerate}


{\bf Bullets} are done like this

\begin{itemize}
\item
First bullet.

\item
Second bullet.

\end{itemize}





\section{Things to Explore}
\label{explore}

Go ahead and move the theorems, definitions and items around.
After compiling twice, the references will be correct.

\begin{itemize}
\item
The first theorem is Theorem \ref{th1}.

\item
The second theorem is Theorem \ref{th2}.

\item
The first definition is Definition \ref{def1}.

\item
The second definition is Definition \ref{def2}.

\item
The first enumeration entry is \ref{en1}.

\item
The second enumeration entry is \ref{en2}.


\item
The an entry in the bibliography is \cite{bib1}.

\item
The another entry in the bibliography is \cite{bib2}.

\end{itemize}


\section{Bad Line and Page Breaks}


Although \LaTeX \ is designed to produce pages that look good,
sometimes it produces a ``widow" (first line of a paragraph
alone at the end of a page) or an ``orphan"
(last line of a paragraph alone at the start of a new page).
To force a page break, use the
$\setminus $clearpage command.


To force a right-justified line break,\linebreak use the
$\setminus $linebreak command.
(But don't spread lines as above. You can see that it looks funny.)



\subsection{Long Subsection Headings 
Long Subsection \\
Headings 
Long Subsection Headings
}


Sometimes a section or subsection heading can be a bit long.
To accommodate width requirements for headings, 
use hard carriage returns (``$\setminus \setminus $")
to insert line breaks in the heading. These hard carriage returns 
will also be used in the table of contents, but 
I was told that that is acceptable. 
Remember to set up the heading in inverted pyramid form
(each line is narrower than the previous line).


