\thispagestyle{empty}


\begin{singlespace}

\centerline{\bf \Large Students: Remember This}



\begin{itemize}
\item
When printing from a pdf, you {\bf must select the option
``Page scaling: None."} Otherwise Adobe will be ``helpful" and
mess up all margins. Printing from a dvi or printing pdfs
at office services is safe.

\item
To create a pdf, use a dvi$\to $pdf conversion, not pdf\LaTeX, which is a
different compiler.


\item
Check for ``widows," ``orphans," and words that spill into the margin.
These should be the most common formatting errors that \LaTeX \
does not fix automatically.

\item
Format checkers will mark English mistakes if they happen to see them,
but they are not proofreaders. So passing the format check does not
mean your spelling and grammar are correct, and mistakes that were
not caught in one format check may be caught in another.
(The goal for using this template is to
pass the format check on the first try, though.)

\item
Remove this page, but keep the next
page on top of all drafts submitted for
format and English checks.

\item
Remove this page and the next one from the final version
that is submitted for binding.


\end{itemize}


Good advice for feedback from proofreaders in general:
When the format for a certain type of entry, say a section heading,
is marked as ``to be corrected" (long section headings are supposed
to be in inverted pyramid form), it is best to
check the format of {\em all} these entries, as
the format needs to be
consistent overall.


\end{singlespace}




\clearpage

\thispagestyle{empty}

\begin{singlespace}

\centerline{\bf Notes for Proofreading and Format Checking}

\vspace{.1in}


We appreciate the services provided by proofreaders and
format checkers to assure uniformly high quality of documents
produced at Louisiana Tech University.
Because of the differences between mathematical
documents and other documents, we request that the following
be kept in mind.

\begin{itemize}
\item
This document was typeset using Louisiana Tech's approved
\LaTeX \ template. Therefore most formatting
should be a non-issue.

\item
Issues that should not require correction, except
as indicated below.

\vspace{.1in} %counteracts the macro that is calibrated for double spaced typing

\begin{itemize}
\item
Global margins, order of sections, page numbering,
title page, format of
headings, table of contents, list of figures, list of tables.
(All approved when the template was created.)

\item
\LaTeX \ is {\em the} professional standard for mathematical typesetting.
Equations are typeset with
the \LaTeX \ default options, which should not be adjusted.

\item
Some built in font sizes cannot be changed.
Font sizes for headings, etc.,
were approved, even if they may look a little different
than in WORD documents.

\end{itemize}

\item
Issues that require attention.
There are some situations in which
\LaTeX' automatic formatting is less than optimal.

\vspace{.1in} %counteracts the macro that is calibrated for double spaced typing

\begin{itemize}
\item
Margin infractions on individual lines.
The global margins have been approved, but if the program
does not know how to split a long term, it can spill into the margin.

This is especially likely in typeset equations and can and should be
fixed.

\item
Widows and orphans. Linebreaking is automatic and sometimes
leaves the first line of a paragraph on the preceding page or
puts the last line on the next page.

\item
English spelling, grammar, punctuation, etc.

\item
{\em Gross} infractions on the placement and spacing of figures.

Because of the way \LaTeX \ imports and creates images,
the distance between a figure and its caption can vary slightly.
Large white spaces should be flagged, though.

\end{itemize}

\item
Please mark {\em all} recommended changes in the first
pass through.

\end{itemize}

\end{singlespace}


\clearpage
